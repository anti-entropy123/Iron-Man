\definecolor{keywordcolor}{rgb}{0.8,0.1,0.5}
\definecolor{webgreen}{rgb}{0,.5,0}
\lstset{language=[AspectJ]Java,
basicstyle=\footnotesize,
keywordstyle=\color{keywordcolor}\bfseries, %\underbar,
identifierstyle=,
commentstyle=\color{blue} \textit,
stringstyle=\ttfamily,
showstringspaces=false,
captionpos=b
}
\chapter{类设计}

\section{Controller设计}

\subsection{AdmController}

1. 功能: 接收管理员端发出的http请求,并返回相应的数据 \\
2. 方法定义
\begin{lstlisting}
    /**
    * 获取全部的符合类型的帖子
    * GET /api/adm/getCard/
    */
    public GetCardResponse getCards(
        int type, 
        String userId, 
        Long minDate,
        Long maxDate, 
        int page);
    /**
    * 删除指定帖子
    * DELETE /api/adm/deleteCard/
    */
    public DefaultResponse deleteCards(
        DeleteCardsRequest deleteCardsRequest);
    /**
     * 获取帖子详情
     * GET /api/adm/getCardDetail/
     */
    public CardResponse getCards(
        int type,
        int page,
        String location,
        double minPrice,
        double maxPrice,
        double minSquare,
        double maxSquare,
        int unitType,
        boolean hasHouseResource);

  /**
   * 获取符合条件的用户信息
   * GET /api/adm/getUser/
   */
    public GetUserResponse getUser(
        String nickname,
        String userId,
        String mobileNumber,
        int page);

  /**
   * 删除指定用户
   * DELETE /api/adm/deleteUser/
   */    
    public DefaultResponse deleteUser(
        DeleteUserRequestBody deleteUserRequestBody);
    \end{lstlisting}



\subsection{BaseController}
1.功能:接收并响应发送验证码、绑定手机号请求 \\
2.方法定义
\begin{lstlisting}
    /**
    * 向用户发送验证码
    * POST /api/bas/getCheckCode/
    */
   public DefaultResponse getCheckCode(
           MobileNumber mobileNumber);
   /**
    * 为用户绑定手机号
    * POST /api/base/bindMobile/
    */
   public DefaultResponse bindMobile(
           MobileAndCheckCode mobileAndCheckCode,
           HttpHeaders headers);
\end{lstlisting}

\subsection{CommentController}
1.功能:接收并响应获取评论、发送评论、删除评论的请求 \\
2.方法定义
\begin{lstlisting}
    /**
    * 获取指定帖子的评论
    * GET /api/comment/getComments/
    */
   public GetCommentResponse getComments(
           String cardId,
           int pageNum);
   /**
    * 获取指定评论的回复
    * GET /api/comment/getReplies/
    */
   public GetRepliesResponse getReplies(String commentId);
   
   /**
    * 上传评论
    * POST /api/comment/addComment/
    */
   public DefaultResponse addComment(
           AddCommentRequest addCommentRequest);
   
   /**
    * 删除指定评论
    * DELETE /api/comment/deleteComment/
    */
   public DefaultResponse deleteComment(
           DeleteCommentRequest deleteCommentRequest;
\end{lstlisting}

\subsection{DetailController}
1.功能:接收并响应获取帖子详情、提出个人申请、提出合租申请、房主查阅申请的个人和队伍、处理申请的请求 \\
2.方法定义
\begin{lstlisting}
    /**
    * 获取帖子详情
    * GET /api/detail/getCardDetail/
    */       
   public DetailCardResponse getCardDetail(String cardId);
   
   /**
    * 提交个人发出的申请
    * POST /api/detail/orderApply/
    */
   public DefaultResponse orderApply(OrderApplyRequest orderApplyRequest);


   /**
    * 提交组队发出的申请
    * POST /api/detail/orderTeamApply/
    */
   public DefaultResponse orderTeamApply(
           OrderTeamApplyRequest orderTeamApplyRequest);
   
   /**
    * 房主获取个人或队伍对自己发出的申请
    * GET /api/detail/getApply
    */
   public GetApplyResponse getApply(String cardId);
   /**
    * 处理申请
    * POST /api/detail/processApply/
    */
   public DefaultResponse processApply(
             ProcessApplyRequest processApplyRequest);
\end{lstlisting}


\subsection{HomeController}
1.功能:接收并响应获取符合条件的帖子的大致信息的请求 \\
2.方法定义
\begin{lstlisting}
    /**
    * 获取符合条件的帖子的大致信息
    * GET /api/home/getCards
    public CardResponse getCards(
              int type,
              int page,
              String location,
              double minPrice,
              double maxPrice,
              double minSquare,
              double maxSquare,
              int unitType,
              boolean hasHouseResource);
\end{lstlisting}
\subsection{LoginController}
1.功能:接收并响应用户登录请求 \\
2.方法定义
\begin{lstlisting}
    /**
    * 用户登录
    * POST /api/login/
    */
   public LoginToken commonUserLogin(LoginCode loginCode);
\end{lstlisting}

\subsection{MessageController}
1.功能:接收并响应获取消息列表的请求 \\
2.方法定义
\begin{lstlisting}
    /**
    * 返回指定user的通知消息
    * type = 0 user的card被人申请 返回cardId userId(申请人) applyId
    * type = 1 user的队伍有人申请加入 返回cardId userId applyId
    * type = 2 user的入队申请通过了 返回cardId 下同
    * type = 3 user的队伍有人退出
    * type = 4 user的队伍被解散
    * type = 5 user的申请被处理
    * GET /api/message/getAll/
    */
   public GetAllMessageResponse getMethodName(tring userId) 
\end{lstlisting}

\subsection{OrdController}
1.功能:接收并响应获取已完成的订单和未完成的订单的请求 \\
2.方法定义
\begin{lstlisting}
    /**
    * 获取已完成订单
    * GET /api/complete/all/
    */
   public GetCompleteOrdResponse getCompleteOrd(
           tring userId,
           int page);
   /**
    * 获取未完成订单
    * GET /api/incomplete/all/
    * */
   public GetIncompleteOrdResponse getIncompleteOrd(
           String userId,
           int page);
\end{lstlisting}

\subsection{PersonController}
1.功能:接收并响应获取个人信息和修改个人信息的请求 \\
2.方法定义
\begin{lstlisting}
    /**
    * 获取个人信息
    * GET /api/person/info/
    */
   public GetInfoResponse getInfo(String userId);
   
  /**
    * 修改个人信息
    * GET /api/person/update/
    */
   public DefaultResponse updateInfo(
           UpdateInfoRequest updateInfoRequest);
\end{lstlisting}

\subsection{PublishController}
1.功能:接收并响应发布资源和上传图片的请求 \\
2.方法定义
\begin{lstlisting}
    /**
    * 发布帖子
    * POST /api/publish/
    */
   public DefaultResponse publish(PublishRequest publishRequest);
   
  /**
    * 发布图片
    * POST /api/publish/image/
    */
   public PublishImageResponse postMethodName(MultipartFile image );
\end{lstlisting}

\subsection{TeamController}
1.功能:接收并响应获取合租队列、新建合租队伍、加入合租队伍、删除推出队伍的请求 \\
2.方法定义
\begin{lstlisting}
    /**
    * 获取队伍列表
    * GET api/union/getTeam/
    */
   public GetTeamResponse getTeam(String cardId,HttpHeaders headers);
   
   /**
    * 创建队伍
    * POST api/union/createTeam/
    */
   public DefaultResponse createTeam(
           CreateTeamRequest createTeamRequest);
   /**
    * 加入队伍
    * POST api/union/joinTeam/
    */
   public DefaultResponse joinTeam(JoinTeamRequest joinTeamRequest);

   /**
    * 删除或退出队伍
    * POST api/union/leaveTeam/
    */
   public DefaultResponse leaveTeam(LeaveTeamRequest leaveTeamRequest);
\end{lstlisting}


\section{Service设计}
\subsection{AdmCardService}
1. 功能 :为管理员请求提供服务 \\
2. 方法设计
\begin{lstlisting}
    /**
     * 按照条件返回帖子内容
     */
    public GetCardResponse.Data[] getCardsByRequire(
        int type,
        String userId,
        long minDate,
        long maxDate,
        int page); 

    /**
     * 返回符合条件的帖子的页数
     */
    public int getCardsPagesByRequire(
        int type,
        String userId,
        long minDate,
        long maxDate,
        int page); 

    /**
     * 删除指定帖子
     */
    public void deleteCards(String[] cardIds);

    /**
     * 返回符合条件的用户信息
     */
    public GetUserResponse getUser(
        String nickname,
        String userId,
        String mobileNumber,
        int page);

    /**
     * 删除指定用户
     */
    public DefaultResponse deleteUser(String[] userIds);
\end{lstlisting}
 
\subsection{BaseService}
1. 功能 :为BaseController提供服务 \\
2. 方法设计
\begin{lstlisting}
    /**
     * 获取openId
     */
    private String parseJsonToOpenId(String json);


    /**
     * 为用户记录登录状态
     */
    public LoginToken login(String code);

    /**
     * 注册新用户
     */
    public void register(String openId);


    /**
     * 发送验证码到指定手机号
     */
    public DefaultResponse getCheckCode(String mobileNum);


    /**
     * 绑定手机号 判断验证码是否与发出的相同
     */
    public DefaultResponse bindMobile(String userId, String checkcode, String mobileNum);
\end{lstlisting}

\subsection{CardService}
1. 功能 :为PublishController提供帖子相关的服务 \\
2. 方法设计
\begin{lstlisting}
    /**
    * 保存上传的帖子
    */
   public boolean publishCard(PublishRequest publishRequest);
\end{lstlisting}

\subsection{CommentService}
1. 功能 :为CommentController提供服务 \\
2. 方法设计
\begin{lstlisting}
    /**
    * 获取指定帖子的评论
    */
   public GetCommentResponse getComments(String cardId, int pageNum);

   /**
    * 获取指定评论的回复
    */
   public GetRepliesResponse getReplies(String commentId);

   /**
    * 添加评论
    */
   public DefaultResponse addComment(AddCommentRequest addCommentRequest);


   /**
    * 删除评论
    */
   public DefaultResponse deleteComment(DeleteCommentRequest deleteCommentRequest);
\end{lstlisting}

\subsection{DetailService}
1. 功能 :为DetailController提供服务 \\
2. 方法设计
\begin{lstlisting}
    /**
    * 获得指定帖子的详情
    */
   public DetailCardResponse getCardDetail(String cardId);


   /**
    * 向指定帖子发出申请
    */
   public DefaultResponse orderApply(String userId, String cardId);

   /**
    * 以合租队伍身份向指定帖子发出申请
    */
   public DefaultResponse orderTeamApply(String userId, String teamId);
   /**
    * 获取申请
    */
   public GetApplyResponse getApply(String cardId);

   /**
    * 处理申请
    */
   public DefaultResponse processApply(String applyId);
\end{lstlisting}

\subsection{HomeService}
1. 功能 :为HomeController提供服务为管理员请求提供服务 \\
2. 方法设计
\begin{lstlisting}
    /**
    * 获取全部的帖子
    */
   public CardResponse getALLCards(int page);


   /**
    * 获取符合条件的帖子
    */
   public CardResponse getCardsWithCondtion(
           int type,
           int page,
           String location,
           double minPrice,
           double maxPrice,
           double minSquare,
           double maxSquare,
           int unitType,
           boolean hasHouseResource);
\end{lstlisting}

\subsection{ImageService}
1. 功能 :为PublishController提供上传图片服务 \\
2. 方法设计
\begin{lstlisting}
    /**
    * 将图片存到云上返回url
    */
   public PublishImageResponse saveImage(MultipartFile image);
\end{lstlisting}

\subsection{MessageService}
1. 功能 :为MessageController提供服务 \\
2. 方法设计
\begin{lstlisting}
    /**
     * 获取发给指定user的消息通知
     */
    public GetAllMessageResponse.Data.Message[] getLogsByAimUser(String userId);
\end{lstlisting}

\subsection{OrderService}
1. 功能 :为OrderController提供服务 \\
2. 方法设计
\begin{lstlisting}
    /**
    * 获取指定完整的订单
    */
   public GetCompleteOrdResponse getCompleteOrd(String userId, int page);


   /**
    * 获取指定未完成订单
    */
   public GetIncompleteOrdResponse getIncompleteOrd(String userId, int page);
\end{lstlisting}

\subsection{TeamService}
1. 功能 :为TeamController提供服务 \\
2. 方法设计
\begin{lstlisting}
    /**
     * 获取指定合租队伍信息
     */
    public GetTeamResponse getTeam(String cardId, String userId);
    /**
     * 创建一个合租队伍
     */
    public DefaultResponse createTeam(String userId, String cardId, int maxNum);
    /**
     * 加入一个合租队伍
     */
    public DefaultResponse joinTeam(String teamId, String userId);
    /**
     * 退出或删除一个合租队
     */
    public DefaultResponse leaveTeam(String teamId, String userId);
\end{lstlisting}

\section{Dao设计}

\subsection{ApplicationDao}

1.功能:对application和team\_application表的操作 \\
2.方法设计
\begin{lstlisting}
    public String createApplication(CardApplication app);

    public String createApplication(TeamApplication app);

    /**
     * 修改申请的状态为完成
     */
    public boolean processApplication(TeamApplication app); 
    public boolean processApplication(CardApplication app); 


    /**
     * 查询发送给某用户的所有申请
     */
    public Application[] queryApplicationsByUserId(String targetUserId);


    /**
     * 根据 cardId 查询该card对应的有关的CardId
     * @param cardId 理论上 cardId 只能是租房的card的id
     * @return Application[] 中的对象的实际类型应该是CardApplication
     */
    public Application[] queryCardApplicationsByCardId(String cardId);


    /**
     * 根据 appId 查询对应的申请
     * @param ApplicationId
     * @return 应返回真实类型, 并向上转型
     */
    public Application queryApplicationByAppId(String applicationId);


    /**
     * 根据 UserId查询由该 User 发起的所有 CardApplication
     * @param applicantId 申请的发起者的Id
     * @return    CardApplication
     */
    public CardApplication[] queryCardAppByApplicantUserId(String applicantId);
\end{lstlisting}

\subsection{CardDao}
1.功能:对card表和card\_coordinate的操作\\
2.方法设计
\begin{lstlisting}
    /**
    * 
    * @param cardId
    * @return
    */
   public Card queryCardByCardId(String cardId);


   /**
    * 查询某用户发起的所有帖子
    * @param userId
    * @return
    */
   public Card[] queryCardsByUserId(String userId);


   /**
    * 查询该用户发起的所有已完成的帖子
    * @param userId
    * @return
    */
   public Card[] queryFinishCardsByUserId(String userId);


   /**
    * 获取所有类别的最新帖子, 按页返回, 可以不对坐标进行初始化
    * @param page
    * @return 
    */
   public Card[] queryALLCards(int page);


   /**
    * 出租贴
    * @param page 
    * @param location 不需要时为 null
    * @param priceRange 价格区间, 不需要时为null
    * @param unitType 户型, 不需要时为 -1
    * @return
    */
   public RentCard[] queryRentCards(int page, String location, Double[] priceRange, Double[] squares, int unitType);
   /**
    * 
    * @param page
    * @param location
    * @param priceRange
    * @param unitType
    * @return
    */
   public AskRentCard[] queryAskRentCards(int page, String location, Double[] priceRange, Double[] squares, int unitType);
   /**
    * 
    * @param page
    * @param location
    * @param priceRange
    * @param unitType
    * @return
    */
   public SellCard[] querySellCards(int page, String location, Double[] priceRange, Double[] squares, int unitType);
   /**
    * 
    * @param page
    * @param location
    * @param priceRange
    * @param unitType
    * @return
    */
   public AskSellCard[] queryAskSellCards(int page, String location, Double[] priceRange, Double[] squares, int unitType);
   
   /**
    * 
    * @param page
    * @param location
    * @param priceRange
    * @param unitType
    * @return
    */
   public RoomMateCard[] queryAskRoomMateCards(int page, String location, Double[] priceRange, Double[] squares, int unitType, Boolean hasHouseResource);
   
   
   /**
    * 新建一个帖子
    * @param card
    * @return
    */
   public boolean createCard(Card card);
   
   /**
    * 根据cardId更新数据
    * @param card
    * @return
    */
   public boolean updateCard(Card card);


   /**
    * 根据cardId, 将帖子状态更新为已完成
    * @param cardId
    * @return
    */
   public boolean finishCard(String cardId);


   /**
    * 为某个用户记录未完成的订单
    * @param userId
    * @param cardId
    * @return
    */
   public boolean recordUncompleteCardForUser(String userId, String cardId);


   /**
    * 删去某用户未完成订单的记录
    * @param userId
    * @param cardId
    * @return
    */
   public boolean deleteUncompleteCardForUser(String userId, String cardId);


   /**
    * 当前端请求查询用户参与的未完成的订单时, 通过此接口查询用户所参与的未完成订单(不包括自己发起的).
    * @param userId
    * @return
    */
   public Card[] queryUncompleteCardsByUserId(String userId);


   /**
    * 将帖子的状态设置为完成
    * @param cardId
    */
   public void setStatusTrue(String cardId);


   /**
    * 返回带有坐标的Card数组
    * @return
    */
   public Card[] queryCardsWithCoordinates();
\end{lstlisting}

\subsection{CommentDao}
1.功能:对comment表的操作  \\
2.方法设计
\begin{lstlisting}
    /**
    * 按页获取某帖子下主楼评论
    * @param cardId
    * @return 所有主楼评论(不含回复)
    * 9.11 测试通过
    */
   public Comment[] queryCommentsByCardId(String cardId, int page);
   /**
    * 获取某主楼评论下的所有回复
    * @param commentId
    * @return
    */
   public Reply[] queryRepliesByCommentId(String commentId);
   
   /**
    * 
    * @param comment
    * @return
    */
   public boolean createComment(Comment comment);


   /**
    * 删除某主楼评论, 同时删除其所有reply
    * @param commentId
    * @return
    */
   public boolean deleteComment(String commentId);
   
   /**
    * 仅删除某 repley
    * @param commentId
    * @return
    */
   public boolean deleteReply(String commentId);


   /**
    * 查询某个主楼评论的回复数量
    * @param commentId
    * @return
    */
   public int queryReplyNumberByCommentId(String commentId);

   /**
    * 
    * @param commentId
    * @return
    */
   public Comment queryCommentByCommentId(String commentId);
\end{lstlisting}

\subsection{LogDao}
1.功能:对log表的操作\\
2.方法设计
\begin{lstlisting}
    /**
    * 插入一条log
    * @param logEntity
    */
   public void addLog(Log log);


   /**
    * 查询log
    * @param aimUserId
    * @return
    */
   public Log[] queryLogByAimUserId(String aimUserId);


   /**
    * 将log置为已读
    * @param applyId
    */
   public void setTrueByApplyId(String applyId);


   /**
    * 将指定log置为已读
    * @param logId
    */
   public void setTrueByLogId(int logId);
\end{lstlisting}

\subsection{TeamDao}
1.功能: 对team表和team\_member表的操作 \\
2.方法设计
\begin{lstlisting}
    /**
    * 根据(出租)帖子id查此页面所有的队伍
    * @param cardId
    * @return 
    */
   public Team[] getTeamsByCardId(String cardId);


   /**
    * 
    * @param userId
    * @param cardId
    * @return
    */
   public boolean createTeam(Team team);


   /**
    * 
    * @param userId
    * @param teamId
    * @return
    */
   public boolean addUserToTeam(String userId, String teamId);


   /**
    * 
    * @param userId
    * @param teamId
    * @return
    */
   public boolean deleteUserFromTeam(String userId, String teamId);


   /**
    * 根据teamId查询该team所属的card的id
    * @param teamId
    * @return
    */
   public String queryCardIdFromTeamId(String teamId);


   /**
    * 根据teamId查询该team的队长的id
    * @param teamIds
    * @return
    */
   public String queryCaptainIdFromTeamId(String teamId);


   /**
    * 查询队长id和cardId与参数相符的队伍
    * @param captainId 队长的userId
    * @param cardId    cardId
    * @return          若没有相应的team, 则返回null
    */
   public Team queryTeamByCaptainIdAndCardId(String captainId, String cardId);


   /**
    * 返回该 teamId 对应的 Team
    * @param teamId
    * @return Team 中的 members 字段也应该被初始化
    */
   public Team queryTeamByTeamId(String teamId);


   /**
    * 删除队伍
    * @param teamId
    * @return
    */
   public boolean deleteTeamByTeamId(String teamId);
\end{lstlisting}

\subsection{UserDao}
1.功能:对team表、team\_member表、user表、application表、team\_application表、card表的操作\\
2.方法设计
\begin{lstlisting}
    /**
    * 根据 openId 查用户数据
    * @param openId
    * @return 
    */
   public User queryUserByOpenId(String openId);


   /**
    * 根据teamId, 查询该队伍的队长
    * @param teamId
    * @return 
    */
   public User queryCaptainByTeamId(String teamId);
   
   /**
    * 根据teamId, 查询该队伍下所有成员
    * @param teamId
    * @return 成员列表, 要求队长位于数组之首
    */
   public User[] queryUsersByTeamId(String teamId);


   /**
    * 根据applicationId查询该申请的发起人
    * @param applicationId
    * @return 
    */
   public User queryUserByApplyId(String applicationId);
   /**
    * 根据cardId查帖子的发布者
    * @param cardId
    * @return
    */
   public User queryUserByCardId(String cardId);
   /**
    * 新增用户
    * @param user
    * @return 
    */
   public boolean registerUser(User user);
   
   /**
    * 
    * @param user
    * @return
    */
   public boolean updateUserDataByUserId(User user);


   /**
    * 为某用户绑定手机号
    * @param userId
    * @param mobilePhone
    * @return
    */
   public boolean bindMobileNumberByUserId(String userId, String mobilePhone);


   /**
    * 根据下述条件筛选
    * @param name 如果不需要则为 null
    * @param userId 如果不需要则为 null
    * @param mobileNumber 如果不需要则为 null
    * @return
    */
   public User[] queryUsers(int page, String name, String userId, String mobileNumber);
\end{lstlisting}

\section{实体类设计}
\begin{enumerate}
    \item RequsetBody包:本包内的所有类为对http请求的请求体的封装,格式对应API接口文档内的定义
    \item ResponseBody包:本包内的所有类为对http响应的body的封装,格式对应API接口文档内的定义
    \item entity包:对应数据库表的实体类
    \end{enumerate}
 
\section{Mapper设计}
一个mapper对应一个数据库表,封装对该表的增删查改操作

\section{Utils设计}

\subsection{DateUtil}
1.方法设计
\begin{lstlisting}
    public String getDate()
    
    public String getDateFromLong(long time);
    
    public int getDays(long endTime, long startTime);
    
    public int getDays(String endTime, String startTime);
    public boolean isLaterThanCurrentStartTime(long startTime);

\end{lstlisting}

\subsection{HttpRequsetUtil}
1.方法设计
\begin{lstlisting}
    private String sendHttpRequest(String url, String param);
    /**
      *使用code向微信请求获得用户openId
      */    
      public String getOpenIdByCode(String code);
    /**
      *调用外部接口向指定手机发送验证码   
      */
      public boolean sendSMSMessage(
              String mobile,
              String checkCode);
    /**
      *向微信服务器请求access\_token
      */  
      public String getAccessToken();
\end{lstlisting}

\subsection{ImagetUtil}
1.方法设计
\begin{lstlisting}
    /**
    * 将文件上传到腾讯云并获得url
    */
   public String upload(File image);
   
   /**
    * 将MultipartFile转为File
    */
   public File transMultipartFileTofile(MultipartFile image);
\end{lstlisting}

\subsection{JwtTokenUtil}
1.方法设计
\begin{lstlisting}
    private Claims getClaimsFromToken(String token);
    public String[] getIdAndAuthFromToken(String token);

    public String getOpenIdFromToken(String token);

    public String generateToken(String id, String auth);

    String generateToken(Map<String, Object> claims);
\end{lstlisting}

\subsection{WxMessageUtil}
1.方法设计
\begin{lstlisting}
    /**
    * 向指定用户发出通知提醒  
    */ 
    public void sendMessage(WxMessageRequestBody.Data data, String touser);
\end{lstlisting}

 